\documentclass{amsart}
\pagenumbering{gobble}
\usepackage{mathpazo}
\linespread{1.2}
\usepackage{microtype}
\newcommand{\Z}{\mathbb{Z}}

\title{MAS439: Problem set 1 Solutions}


\begin{document}

\section*{Question 1: Elements of $\Z\times\Z$}
\subsection*{What are the units in $\Z\times \Z$}  In general $(a,b)\in R\times S$ is a unit if and only if $a$ is a unit in $R$ and $b$ is a unit in $S$.  This is because $(a,b)$ is a unit if and only if there is a $(c,d)$ with $(a,b)\cdot (c,d)=1_{R\times S}$.  But this is if and only if $a\cdot c=1_R$ and $b\cdot d=1_S$, i.e. if $a$ is a unit in $R$ and $b$ is a unit in $S$.

Since the only units in $\Z$ are $\pm 1$, we see that there are four units in $\Z\times \Z: (1,1), (1,-1), (-1,1), (-1,1)$.

\subsection*{What are the nilpotent elements of $\Z\times \Z$} In general, an element $(a,b)\neq (0,0)\in R\times S$ is nilpotent if and only if both $a$ and $b$ are nilpotent or zero.  This is because $(a,b)$ is nilpotent if and only if $(a,b)^n=(0,0)$ if and only if $a^n=0$ and $b^n=0$, i.e., if and only if both $a$ and $b$ are nilpotent or zero.

Thus, since $\Z$ has no nilpotent elements, $\Z\times \Z$ has no nilpotent elements.

\subsection*{What are the zero divisors in $\Z\times\Z$} Suppose that $(a,b)\neq (0,0)\in\Z\times\Z$ is a zero divisor.  This happens if and only if there is a $(c,d)\neq (0,0)\in\Z\times\Z$ so that $(a,b)\cdot (c,d)=0$, i.e. $ac=0$ and $bd=0$.  Since $\Z$ has no zero divisors, $ac=0$ if and only at least one of $a,c=0$.  Similarly, at least one of $b,d=0$.  Since by assumption, both of $a,b$ cannot be zero, and both of $c,d$ cannot be zero, we see there are two possibilities: $a=0$ and $d=0$, or $b=0$ and $c=0$.  Thus, the zero divisors of $\Z\times\Z$ are those elements of the form $(a,0)$ or $(0,b)$.

\section*{Question 2: Zero divisors in $R[x]$}
Let $R$ be a nontrivial ring, and let $f=a_nx^n+a_{n-1}x^{n-1}+\cdots+a_0\in R[x]$.  Prove that if $a_n$ is not a zero divisor in $R$, then $f$ is not a zero divisor in $R[x]$.

{\bf Proof:} Suppose that $f=a_nx^n+a_{n-1}x^{n-1}+\cdots+a_0\in R[x]$ were a zero divsior; we must show that $a_n$ is a zero divisor.  Then there exists a $g=b_mx^m+b_{m-1}x^{m-1}+\cdots+b_0\in R[x]$ with $fg=0$.  Note that we can take $b_m\neq 0$.  But expanding the product $fg$ out, we have
$$a_nb_mx^{n+m}+(a_nb_{m-1}+a_{n-1}b_m)x^{n+m-1}+\cdots+(a_1b_0+a_0b_1)x+a_0b_0=0$$
For this to be true, we need the coefficient of every $x^k$ to be equal to zero in $R$; in particular, the coefficient of $x^{n+m}$, namely $a_nb_m$, must be equal to 0.  Since $b_m\neq 0$, this means $a_n$ is a zero divisor.

\section*{Question 3: Counting homomorphisms}
How many different homomorphism $\varphi:R\to S$ are there when:
\subsection*{$R=\Z$ and $S=\Z[x]$} Lemma 4.4 in the notes states that for any ring $S$ there is a unique homomorphism $\varphi:\Z\to S$.  Hence, there is one homomorphism from $\Z$ to $\Z[x]$.

\subsection*{$R=\Z/7\Z$ and $S=\Z/49\Z$} Following the proof of Lemma 4.4, we see that since $\Z/7\Z$ is generated by $1$, there can be at most one homomorphism: $\varphi(1)=1$ and the fact that $\varphi$ preserves addition quickly gives $\varphi([n]_7)=[n]_{49}$.  But this is not a homomorphism, since $[0]_7=[3]_7+[4]_7$ but $$\varphi([3]_7)+\varphi([4]_7)=[3]_{49}+[4]_{49}=[7]_{49}\neq [0]_{49}=\varphi([0]_7).$$  So there are no homomorphisms from $\Z/7\Z$ to $\Z/49\Z$.

\subsection*{$R=\Z/14\Z$ and $S=\Z/7\Z$} The same argument as the previous part shows there can be at most one homomorphism, specicially $\varphi([n]_{14})=[n]_7$.  We claim this actually is a homormorphism.  There's a lot of additions and multiplications to check, so it's easier to shift points of view and think of $\Z/n\Z$ to be the set of cosets of multiples of $n$; that is, elements of $\Z/n\Z$ are the subsets of the form:
$$r+n\Z=\left\{x\in \Z : x=r+nk, \text{ for some } k\in\Z\right\}$$
The homomorphism $\varphi:\Z/14\Z\to\Z/7\Z$ then can be written $\varphi(k+14\Z)=k+7\Z$.  But one then sees:
$$\varphi(k+14\Z)+\varphi(\ell+14\Z)=k+7\Z+\ell+7\Z=k+\ell+7\Z=\varphi(k+\ell+14\Z)$$
\begin{equation*}
  \begin{split}
    \varphi(k+14\Z)\cdot\varphi(\ell+14\Z)&=(k+7\Z)\cdot(\ell+7\Z)\\
    &=k\ell+7k\Z+7\ell\Z+49\Z\\
    &=k\ell+7\Z\\
    &=\varphi((k+14\Z)\cdot (\ell+14\Z))
  \end{split}
  \end{equation*}
\subsection*{$R=\Z\times\Z$ and $S=\Z/12\Z$} This question is more difficult because knowing that $\varphi(1)=1$ and that $\varphi$ preserves addition no longer determines $\varphi$.  Let $\varphi( (1,0))=a$, $\varphi((0,1))=b$.

Suppose that $\varphi$ is a homomorphism.  We see that asking $\varphi$ to be an additive group homomorphism is equivalent to asking $\varphi(x,y)=xa+by$, and so $a$ and $b$ determine $\varphi$.  But not all choices of $a, b\in\Z/12\Z$ will result in $\varphi$ being a ring homomorphism.  To ask that $\varphi$ preserves the identity, is asking $\varphi(1,1)=a+b=1$.

Now suppose that $\varphi$ preserves multiplication.  Since $(1,0)^2=(1,0), (0,1)^2=(0,1)$, and $(1,0)\cdot(0,1)=0$, we see we must have $a^2=a, b^2=b, ab=0$. We claim that if $a, b$ satisfiy these three equations, then $\varphi$ preserves multiplication:
\begin{equation*}
\begin{split}
  \varphi((x,y))\cdot\varphi((v,w))& =(ax+by)\cdot (av+bw)\\
  &=a^2xy+ab(xw+vy)+b^2yw\\
  &=axv+byw=\varphi(xv,yw)=\varphi((x,y)\cdot (v,w))
  \end{split}
\end{equation*}
The only elements $a\in\Z/12\Z$ with $a^2=a$ are $0,1,4,9$, and this together with $a+b=1, ab=0$ means there are four homomorphisms $\varphi_i:\Z\times\Z\to\Z/12\Z$:
$$\varphi_1(x,y)=x \quad \varphi_2(x,y)=y \quad \varphi_3(x,y)=4x+9y\quad\varphi_4(x,y)=9x+4y$$

\end{document}
