\documentclass{beamer}

\usepackage{amsmath, amssymb, hyperref, graphics, wasysym, tikz}
\usetikzlibrary{cd}
\usepackage{mathpazo}

\newcommand{\C}{\mathbb{C}}
\newcommand{\Z}{\mathbb{Z}}
\newcommand{\R}{\mathbb{R}}
\newcommand{\Q}{\mathbb{Q}}
\newcommand{\A}{\mathbb{A}}
\title{MAS439 Lecture 19 \\ Fields of fractions, Function fields}

\date{October 26th}

\begin{document}

\begin{frame}
\titlepage
\end{frame}

\begin{frame}{Going back to Primary school}
    Remember how we construct the rational numbers $\Q$ out of the integers $\Z$:
    \begin{itemize}
    \item A rational number is an equivalence class of $(a,b)\in\Z\times\Z$, with $b\neq 0$ (we write $\frac{a}{b}$)
    \item $\frac{a}{b}\sim \frac{c}{d}$ if $ad-bc=0$
    \item $\frac{a}{b}+\frac{c}{d}=\frac{ad+bc}{bd}$
    \item $\frac{a}{b}\cdot \frac{c}{d}=\frac{a\cdot c}{b\cdot d}$
\end{itemize}
These operations make $\Q$ into a field. The inverse of $a/b$ is $b/a$, and $\Z$ sits inside $\Q$ as a subring, via $n\mapsto n/1$.

But we can generalize this...    
\end{frame}

\begin{frame}{The main definition}
Let $R$ be an \alert{integral domain}.  The \emph{field of fractions} of $R$, denoted $K(R)$, is the field where the elements consists of equivalence classes of the form $a/b$, with $a,b\in R, b\neq 0_R$ with  
\begin{itemize}
\item $\frac{a}{b}\sim \frac{c}{d}$ if $ad-bc=0$
    \item $\frac{a}{b}+\frac{c}{d}=\frac{ad+bc}{bd}$
    \item $\frac{a}{b}\cdot \frac{c}{d}=\frac{a\cdot c}{b\cdot d}$
\end{itemize}


One has to check that $K(R)$ actually is a ring, and a field, which is a bit laborious but not difficult or enlightening.
\end{frame}

\begin{frame}{Examples}
  \begin{block}{$K(\Z)=\Q$}
    This is the example we used to motivate the definition of $K(R)$.
  \end{block}
  \begin{block}{Rational functions}
    Recall that a function of one variable is called \emph{rational} if it is the ratio of polynomials, e.g.
    $$\frac{x}{x^2-1}, \quad \frac{x^3+3x}{ix^2+x+.5}, \frac{x^2-x}{x-1}, \quad \frac{x}{1}$$

    It is clear the set of rational functions form a field, sometimes denoted $\Q(t), \R(t), \C(t)$.
    Then $\Q(t)=K(\Q(t))$.
    \end{block}
\end{frame}
\begin{frame}{More examples - nonisomorphic rings with isomorphic fields of fraction}
  \begin{block}{$K(k)\cong k$}
    If $R$ is already a field, then $K(R)\cong R$.  So, we have $K(\Z)\cong \Q\cong K(\Q)$.
  \end{block}
  \begin{block}{But many more...}
    But maybe we're in between somewhere; let $S=\Z[x]/(2x-1)$.  Then in $S$, we see that $x$ is acting as $1/2$, and in particular we see powers of two have inverses in $S$.   Then
    $K(S)\cong\Q$. 
  \end{block}

  \end{frame}

\begin{frame}{Why integral domains?}
  In the definition of $K(R)$, we required that $R$ be an integral domain.  Why?

  \begin{lemma} Suppose that $R$ is not an integral domain.  Then $K(R)$ is the trial ring.\end{lemma}
  \begin{proof}
    Let $uv=0$ in $R$ with $u,v\neq 0$.  Then in $K(R)$, we have
    $$1=\frac{1}{u}\cdot u\cdot v\cdot \frac{1}{v}=0.$$
    \end{proof}
Even if $R$ isn't an integral domain, we can still add inverses to \emph{some} things...
\end{frame}
  

\begin{frame}{Function Fields}

  \begin{definition} Let $X\subset \A^n_k$ be an affine variety, so by definition $k[x]$ is an integral domain.  The \emph{function field} $k(X)$ of $X$ (sometimes called the \emph{coordinate field}), is the field of fractions of the coordinate ring of $X$, i.e.
    $$k(X):= K(k[X])$$
  \end{definition}

  \begin{example} For $X=\A^n_k$, we have $k[\A^n_k]=k[x_1,\dots, x_n]$, and so $k(\A^n_k)=k(x_1,\dots, x_n)$, the ring of rational functions in $n$ variables.
    \end{example}
\end{frame}

\begin{frame}{More examples: varieties with isomorphic function fields}
  \begin{example} Let $X=V(xy-1)\subset \A^2_\C$.  Then we see that
    $$\C[X]=\C[x,y]/(xy-1)\cong \C[x,x^{-1}]$$
    I.e., the coordinate ring of $X$ is the ring of \emph{Laurent polynomials}.  Thus:
    $$\C(X)\cong \C(t)\cong \C(\A^1_\C)$$
\end{example}
  \end{frame}

\begin{frame}{More examples, continued}
  \begin{example} As a less trivial example, let $Y=V(x^2-y^3)$ be the cuspidal cubic.

    We have seen that $\C[Y]\cong \C[t^2, t^3]$.  But
    $$K(\C[t^2, t^3])\cong \C(t)\cong \C[\A^1_\C]$$
    since $t^3/t^2$ behaves just like $t$.  
     \end{example}
\end{frame}

\begin{frame}{Elements of function fields as functions}
  Remember that we viewed $k[X]$ as polynomial functions from $X\to k$.

  Thus, if $f=g/h\in k(X)$ we may view $f$ as a partially defined function from $f\to k$, where if $a\in X$, we set $f(a)=g(a)/h(a)$, which makes sense as long as $h(a)=0$.

  We write $f:X\dashrightarrow k$ to distinguish that the function may not be defined everywhere.
  \begin{block}{Switching representatives}
    In first and second year, we viewed $\frac{x^2-1}{x-1}$ and $x+1$ as different functions, as the first isn't defined at $x=1$.  But in $\C(x)$, these two functions are defined to be equal.  
  \end{block}
  \end{frame}
\begin{frame}{Domains of rational functions}
\begin{definition}
  Let $f\in k(X)$ be a rational function.
  \begin{itemize}
  \item We say $f$ is \emph{regular} at $a\in X$ if \alert{there exists} a representative $g/h$ of $f$ with $h(a)\neq 0$.
  \item The \emph{domain} of $f$, written $\text{dom}(f)$, is the set of points $U\subset X$ where $f$ is regular.
  \end{itemize}
  \end{definition}
  Thus, a rational function $f\dashrightarrow k$ gives an actual function $f:\text{dom}(f)\to k$.
  \end{frame}

\begin{frame}{Example of finding the domain}
  Let $X=V(y^2-x^3)$, and consider $f=y/x\in \C(X)$.  It is clear that $f$ is well defined everywhere except perhaps at the origin $(0,0)$.  We now show $(0,0)\notin \text{dom}(f)$.

  Suppose otherwise; then $f$ has a representative $p/q$, with $q(0,0)\noteq 0$. Hence, $q$ has a nonzero constant term, say $q(0,0)=c$.

  But since $y/x\sim p/q$, we have $yq(x,y)=xp(x,y)$.  But the $y$ term of the left hand side is $cy\neq 0$, while the right hand side has no $y$ term, contradiction.
  
  \end{frame}

\begin{frame}{There may not be one best representative}
  Consider $X=V(xy-zw)\subset \A^4_\C$, sometimes called the \emph{conifold}.

  Note that in $\C(X)$ we have $z/x=y/w$.

  Now $z/x$ is well defined as long as $x\neq 0$, while $y/w$ is well defined as long as $w\neq 0$.

  \end{frame}

\begin{frame}{Regular functions}
  \begin{definition}
    A rational function $f\in k(X)$ is called \emph{regular} if its domain of definition $\text{dom}(f)$is all $X$.
  \end{definition}

  \begin{lemma}Let $k=\overline{k}$.  Then $f\in k(X)$ is regular if and only if we have $f\in k[x]\subset k(X)$; that is, if $f$ has a representative of the form $f=g/1$.
  \end{lemma}

  \begin{proof}
    Idea:\begin{itemize}
    \item Show that $I(f)=0\cup\{h: \exists g/h=f\}$ is an ideal in $k[x]$.
    \item Apply the Nullstellensatz to $I(f)$.
    \end{itemize}
  \end{proof}
  \end{frame}
    

\end{document}
